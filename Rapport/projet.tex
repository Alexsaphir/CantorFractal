\documentclass[a4paper, 12pt]{report}

\usepackage[utf8]{inputenc} % un package
\usepackage[T1]{fontenc}      % un second package
\usepackage[frenchb]{babel}
\usepackage{graphicx}
\usepackage{amsthm}
\usepackage{amsfonts}
\usepackage{amssymb}
\usepackage{amsmath}
\usepackage{ textcomp }




\headheight=0mm
\topmargin=-10mm
\oddsidemargin=-1cm
\evensidemargin=-1cm
\textwidth=18cm
\textheight=25.5cm
\parindent=0mm

\title{ \bf Fractales et dimension de Haussdorf}
\author{Mechineau Alexandre}

\makeindex
\begin{document}
\maketitle
\newtheorem{definition}{Definition}
\tableofcontents


\chapter{\bf Introduction sur les fractales}


Le mot fractale a été introduit par Benoît Mandlebrot en 1974 pour décrire des objet dits brisés ou irrégulier, à la difference des objets géométrique classique tels que les droites, les sphères,...
Cepandant, ces objets avaient déjà été ``découvert'' et cela dès l'antiquité par Apollonius de Perge avec la figure nommé ``la baderne d'Apollonius''\\
Image\\
On remarque que si l'on agrandit une zone de la fractale, le meme motif se repète. C'est une propriété des fractales que l'on verra par la suite.\\
Par la suite, en 1520, Dürer dessine une fractale qui sera nommé Pentagone de Dürer.\\
îmage\\

Il existe trois maniere de definir une fractales :
\begin{enumerate}\itemsep2pt
	\item Par un systeme de fonction itéré(IFS)
	\item Par  une relation de recurrence
	\item Par des processus stochastiques et non déterministes
\end{enumerate}

Dans ce projet, je vais étudier les fractales de type IFS. C'est a dire que  je définit un ensemble de départ E et j'applique mon ensemble de fonction définissant ma fractale sur E.
L'ensemble des points générés converge alors vers la fractale désirée.

Je vais donc, alors étudier dans un premier temps les fonctions definissant une fractale et l'ensemble de départ.
Puis, dans un second temps, j'étudierais la {\bf dimension de Haussdorf} permettant de calculer la dimension d'une fractale.
Enfin, je présenterais de maniere succinte,comment j'ai écrit le programme permetaant de tracer des fractales.

\chapter{\bf Défintions des fractales}
\begin{definition}
	Un objet auto-similaire (self-similarity) est un objet qui conserve sa forme qu'elle que soit l'échelle. 
\end{definition}


\begin{definition}
	Une fractale est un ensemble auto-similaire.
\end{definition}

\begin{definition}
	
	  Soit $P\in\mathbb{R}^n$, soit F: $\mathbb{R}^n\longrightarrow \mathbb{R}^n\times\ldots\times\mathbb{R}^n$,
	      $P\longmapsto\begin{pmatrix}
	      f_{1}(P)	 \\
	      f_{2}(P)	 \\
	      \vdots \\
	      f_{i}(P) 
	      \end{pmatrix}$ 
	      est contractante $\Longleftrightarrow \forall j \in $\textlbrackdbl 1,i \textrbrackdbl$, \quad f_j:\mathbb{R}^n \longrightarrow \mathbb{R}^n$ est contractante
	
	
\end{definition}

\begin{definition}
	Soit S une similitude alors l'image d'un segment est un segment.
\end{definition}


\begin{proof}
 Soit S une similitude de rapport k, soit AB un segment A,B$\in\mathbb{R}^n$ muni de la distance d alors S(A)=A' et S(B)=B'.\\
 On a donc d(A',B')=k*d(A,B).\\
 La longueur du segment A'B' est proportionnelle à la longueur de AB.\\
 Donc les deux segments sont semblables.
\end{proof}

\begin{proof}[Proof of important theorem]
Here is my important proof
\end{proof}

\end{document}


\grid
